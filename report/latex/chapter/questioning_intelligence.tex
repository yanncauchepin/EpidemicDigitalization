\addcontentsline{toc}{chapter}{Questioning intelligence}
\chapter*{Questioning intelligence}

True knowledge is deep self-knowledge and empathy. It should be the very basis of the individual before being interested in the scientific, literary, etc. fields. This is what distinguishes the human from the machine. Afterwards, it is conceivable that it is often suffering that leads us to know ourselves better, or at least difficulty. That's why we sometimes have to learn to listen to the elderly, who have already suffered. This is one of the contradictions of life : how to advance everyone's self-knowledge while maintaining empathy ? Are we all equal in our capacity for personal introspection ? Some may mention military service... In the individual interest, it would be better to simply promote it properly to individuals so that it is voluntary. Knowledge is above all voluntary. One could be interested in the retransmission of wisdom by the elders ; this would bring dynamism and motivation to retirees. The education system has far too much tendency to shorten or hinder the personal development of young people. Education should even be reviewed in many aspects and supervised children with more benevolence and less in an industrial way. Quite rightly, it would be necessary to consider our entire environment to know ourselves better.\\

The mentalities of individuals are too individualistic in this world. They limit themselves to an external expression, limiting their gaze on their very local surroundings. We are much more interested in how we are perceived than we are ourselves. This is why human errors tend to be repeated, and why the emptiness in each of us persists. We then tend to expand our surroundings, while maintaining the same mentalities; Hence the emergence of social networks and personal data. Does he only have an interest in the over-mediatization of people who just want to show themselves, promote debility and irresponsibility ? It is always surprising to see a greater popularity of this kind of media than for media that develop us more. Nowadays, even journalism has little interest. When will the population look in the right place to fill everyone's personal void ? The world is materialistic and cold.\\

What if we changed mentalities ? This should necessarily involve collective awareness. Otherwise, psychological mismatches would be created that lead to individual misfortunes.\\

What if we all focused on our inner expression ? We would be able to take a step back from the world as a whole. One would be committed to responding to the real problems of humanity. Respect and compassion could be further promoted to all. Consider more of each other as equals. It would remove the fear of individuals to limit themselves, which often comes from education elsewhere. There would be much less sadness and regret in the eyes of the dying.\\

We would then be more inclined to really accept humanity and diminish the material and individualistic orientation of the world. One might wonder about the real problems of a human nature, such as psychological suffering, euthanasia or the "financial cost of a human life". Nowadays, empathy is too often emphasized only when suffering comes to us. It is only when we suffer a natural disaster that we wonder where others are. Then there is the "legitimate" penalty which, sooner or later, calls dishonest and malicious people to order...\\

However, whatever one does, entropy is destined to increase over time. And paradoxically, the origin of this entropy is substantially linked to the search for the idea of happiness proper to each one ; the term "idea" is voluntary, and refers to the distinction with false happiness. But we are leaving the context of artificial intelligence.\\

The greatest strength of entropy are human weaknesses.\\
The danger: giving power to the expansion of entropy.\\

It is well known that the most expensive resource of tomorrow will be data. The tool will be artificial intelligence. Two processes can be distinguished in data processing :\\
\begin{itemize}
\item Or by the judgment of the individual through his personal data. We represent the individual with numbers, and we try to predict his factual behavior or classify him. The individual is restricted to his nature as a machine. But no artificial intelligence will be able to understand the human soul. This is why the relevance and processing of personal data is highly controversial.
\item Or by analyzing the environment by global data. We model the environment and work on human phenomena as a whole. This requires a design effort to link the data together, like a black box. But this treatment would be in respect of the human condition.\\
\end{itemize}

For example, it is likely that few Europeans would want to be subjected to the Chinese policy of surveillance of its population by their personal data. We can also cite the Cambridge Analytica affair which influenced the free will of individuals in the election of the president of one of the most powerful countries in the world. Their use is widespread at the scale of commercial enterprises. Is the interest therefore purely economic ? See a question of control of individuals ?\\

It is utopian, and even dangerous, to believe that an artificial intelligence can solve everything on its own, without human introspection and empathy. Human knowledge is necessary.\\

It is mandatory to respect each other's experience, and often to put oneself in the other's shoes. But it already seems irrational to believe that any individual can appropriate the personality and tolerate the choices of all. The justice of our world is not truly just and omniscient. A bit like religions, however, we can find at least one form of guide, which directs individuals towards rational and balanced
decisions. Even if it too often takes us away from our personal introspection...\\

Too often ideas are pointed out to explain and give an excuse to malevolence. We can talk about overpopulation, natural selection, discrimination, cultural or religious hatred, etc. But wouldn't that be a lack of intelligence ? An inability to solve problems and a fatalistic side ?\\

The world is too complex to uphold divine justice.\\
So why not anticipate disasters as a whole ? Environmental, epidemic, etc.\\

A simple example : take the case of the giant Amazon, which got rich during the Covid-19 pandemic. The company does not seem to convey a good benevolence of the individual, either at the employer level or their economic funds. One of their plans would be to have packages delivered by drones. The idea is innovative, it must be emphasized. But quite rightly, there seems to be a danger from a humanitarian point of view. It is not a question of rejecting the idea. But why not invest this innovative technology in humanitarian first ?! For example, it could be used to better manage forest fires. If only to sow a set of flame retardants, and if possible environmentally friendly, around a fire to stop its spread. Too often we see households and individuals suffering from these disasters. We could then focus on communication, logistics, resistance of materials, etc. And many of these technologies would be identical to the original project.\\

The limit : money. Money has a very high force of inertia. Money too often comes from money itself. He rarely has an interest in empathetic projects that consume him. But it is necessary to find a balance that is sometimes difficult to determine, have or maintain. Especially since this balance can always be criticized by the outside. And the greatest difficulty is above all to acquire the financial base to initiate projects, even on the scale of a human lifetime.