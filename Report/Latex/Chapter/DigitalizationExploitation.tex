\chapter{Exploitation of digitalizations}

\section{Digitalization over time}

The challenge of this project is based on the digitalization of the circulation of infectious agents in the mobility of a society. For this, it is necessary to initialize an initial situation of the company and to make it evolve day by day until a certain final point to be defined. The initial state can be obtained by initializing the environment with the various elements mentioned above and contaminate a certain number of individuals, deterministically or randomly. On the other hand, we can stop digitalization under different criteria :\\

\begin{itemize}

\item We can define a predefined number of days to digitize and evaluate the final state of the company. This is one of the important points in the objective of combinatorial optimization: to restrict the research space on time. From an algorithmic point of view, it seems obvious anyway to set a limit beyond which digitalization stops.

\item With a view to optimization, we can establish shutdown conditions according to the situation of the company. To make a parallel with a video game, this could represent conditions of victory of the management. Conversely, we can also set conditions of defeat where the management is not sufficiently optimal to maintain a minimum state of society. Among the conditions for victory, we can identify good health assessments of society, where there is little contamination of infectious agents left. From a utopian point of view, one could say that the victory would be to completely eradicate infectious agents from society. However, this seems unimaginable given the external influence of the region whose sanitary conditions are affected by our management. As for the conditions of defeat, in the event of a serious economic crisis in society, digitalization can be stopped. Similarly, we can add health conditions that are too critical for society, whether in terms of mortality, contamination or available resources.\\

\end{itemize}

A digital observation unit represents the health and economic evolution of society and its individuals over a defined period of time. This period of time is defined as a fixed number of days, which remains the same throughout digitalization. It is at the end of each of its diagnoses that it is appropriate to modify or not the political strategy. From a realistic point of view, this seems all the more lucid because a policy cannot define a day-to-day strategy: it is necessary to take into account the adaptation of individuals. The project must be pragmatic to be interesting.\\

To best diagnose the status of the company, it is wise not to limit yourself to a final health and economic assessment at the end of each observation. Data from each observation can be expanded with daily health and economic assessments. This data is particularly relevant in the event that the days follow a calendar logic and where there is a trend on the dynamics of digitalization. Each observation would thus have a set of daily data to analyze in order to determine the appropriate decisions to be made for the next observation period. These diagnoses could be further explored in combinatorial optimization with reinforcement learning.\\

In practice, the implementation of this digitalization is not so simple to perform. It is necessary to group all the elements of the design mentioned above in a simple, readable and efficient implementation. It is necessary to record the results of the scans over the days. Data memory management can be enriched in static implementations. By associating the number of the day with the hash tables, we can analyze the results of dynamic digitizations, associated with each individual and space. The interest is to observe the times of occupation in the spaces and to evaluate the contaminations as well as the economic stability.\\

In the hash table of the spaces, there is a list of identifiers of individuals sorted according to their arrival time and associated with each space and day of digitization : we then find the identifier of the individuals who occupied the space, the role/importance of this movement as well as their exit time. On the side of the hash table of individuals, we find the updated scenarios by subtracting the duration of digitized journeys: we then find a list sorted with the time of arrivals, the identification of spaces and the time of exits.\\

In addition to this data, digitalization also records the health assessments of individuals in the infectious layer itself. The updating of the infectious states of individuals is also almost autonomous in this same layer. The idea is to decentralize the memory management of data to the appropriate layers. We therefore find the same principle with the management of data concerning territorial decisions.\\

As a general rule, it will always be interesting to keep the data of digitalizations, if only to be able to analyze the weaknesses of the design. We can always ask ourselves the question of the relevance of keeping data. The question is: do we really pretend to believe that the design is sufficient and does not need to be perfected ? However, the more we can trace the process, the more we can reason.\\

Already, a significant perspective of digitalization is the statistical and graphical representation of results. And by "results", it is implicit to extract the intermediate results of digitalization such as the dynamics of society.\\

\newpage

\section{Infectious assessment}

\subsection{Contamination}

The infectious assessment is first calculated at the individual level, within the infectious implementation layer.\\

During digitalization, a susceptible individual accumulates infectious particles when in contact with carrier individuals or when in a particle-laden space. At the end of each day, for each susceptible individual, we can trace a sequence of spaces with data on :\\

\begin{itemize}
\item Occupancy times : precise arrival times, precise departure times.
\item The identifiers of individuals sharing the same spaces during the periods of occupation ; We will then have done a scan to identify the patients carriers among them.
\item The ability to facilitate proximity and contamination of the category of each space.
\item The quantities of particles present in each space when the individual arrives, according to the particle conservation parameter of each infection in a space.
\item The surfaces of each space visited ; This parameter will make it possible to directly nuance the contamination of the individual in a space loaded with particles while remaining pragmatic.\\
\end{itemize}

It is necessary to think more deeply about contamination outside a territory before talking about it. Other data are also interesting to integrate into the daily infectious assessment of an individual :\\

\begin{itemize}
\item Contamination in public transport : in the form of a simplistic score according to the type of transport and the frequency of their use by carrier individuals. We could see to detail this consideration from a more individual point of view.
-\item The number of particles accumulated in each individual over the last few days according to the conservation parameter of each infection.
\item The propensity of each individual, especially internal, to move outside the region. Two phenomena can be observed : the more frequently an individual circulates in the territory, the better we can estimate a random consideration outside ; And the less frequently an individual circulates in the territory, the more absurd it is to find a form of logic in external contamination.\\
\end{itemize}

The individual assessment therefore sums these variables weighted according to a generic formula to obtain an individual contamination score for each infection in the database. When at least one of these contamination scores exceeds the contamination threshold of one of the infections, which may depend on the age of the individual, he becomes a carrier of this infection and changes state to be in the incubation phase. As a reminder, it then becomes "immune" to other infectious agents in the environment as long as it is a carrier. Similarly, in a simplistic version, we are not interested in the case where an individual can be reinfected when he is vaccinated or cured : the individual also becomes "immune" to this infection. Of course, even if the object trace of a deceased individual may persist, it is useless to process their cases.\\

\subsection{Evolution of states}

When the individual assessment function indicates contamination or when the individual changes infectious status during the simulation, the person's infectious reflexes change and modify his mobility. It is necessary to update one's propensities in terms of care, unnecessary travel, etc. Typically, this phase is automatically triggered by the infectious implementation layer and passed indirectly to the dynamic digitization layer. It is agreed to apply a generalist and automatic operation to take into account these updates, on a case-by-case basis for each individual.\\

\subsection{Territorial Assessment Scores}

To summarize the infectious evolutions at the end of each day, it is necessary to implement specific functions where one places one's observation on society. These functions depend on several variables to be weighted according to certain weights. For each infection, the following variables can be distinguished :\\

\begin{itemize}
\item The actual and known quantity or proportion of the susceptible population
\item The actual and known quantity or proportion of the carrier population
\item The amount or proportion of the population cured or vaccinated
\item The amount or proportion of the population seriously ill
\item The amount or proportion of the population that died
\item Immediate change in actual and known number of infections
\item A score on the available margin of care for standard patients
\item A score on the available margin of care for the seriously ill
\item etc.\\
\end{itemize}

These different evaluation scores will make it possible to describe the situation of a company, in a multitude of aspects. We would not limit ourselves to a simple score supposed to represent a pseudo-equilibrium, immutable whatever the conditions. We would then start on real multicriteria optimization.\\

\newpage

\section{Economic evaluation}

\subsection{Enterprises}

The economic evaluation of spaces can be carried out on the basis of the affluence of individuals and characteristic specific to each category of spaces. There is a general evaluation function where variables must be inserted as parameters according to the category of the space. We can thus individualize the economic functioning of each space to have more concrete and applicable optimizations. For example, the evaluation function of an office will be different from a restaurant. And of course, some categories do not even have an economic function, such as public parks for example. In the notion of economic function, we are talking here about the opportunity of space to make the work of an individual productive in his environment. In any case, it would be just as interesting to study the question of economics properly today. Each space is therefore visited by a certain number of individuals, each with a defined role. Some may work in this space and have a financial dependence on it. While others may use it in another personal addiction, or otherwise as a hobby. The economic evaluation of a space therefore depends on data on :\\

\begin{itemize}
\item The category of space and its economic characteristics.
\item The total duration of individuals who have worked in space.
\item The number of individuals employed in space.
\item The area of space.\\
\end{itemize}

Depending on the different types of economic functionality, we can add other data on :\\

\begin{itemize} 
\item The number of people who used/visited the space.
\item The total duration of individuals who worked from home. The question concerning the economic factor of production associated with telework is, however, to be discussed because it is highly subjective.\\
\end{itemize}

We can mention some different economic features such as :\\

\begin{itemize}
\item Offices whose economic dependence can be represented only with the occupation of space by employees. It would then be possible to arrange the management of companies by allowing their employees to telework from home. There is a direct and rather simple correlation between the total working hours of employees and the economic valuation of space.
\item Businesses that have an economic dependence very sensitive to the presence of customers. The more individuals who visit these spaces, the more their economic valuation increases. Of course, this evaluation is also very correlated with the presence of employees who sell their products. The effectiveness of having several employees at the same time may be questioned. But we may be going too far in modeling. In a later version, like teleworking, it could be interesting to add a delivery option where businesses send their products directly to individuals. We will not be interested a priori in the terms of delivery.\\
\end{itemize}

\subsection{Individuals}

It is essential to also look at the economy from the point of view of the individual. We must not forget the objective of this project and not get too much into an individual problem where we seek to optimize everyone's income. The interest here is to evaluate the difference between the initial state of society and of each individual in relation to the states of restrictions. That is to say, we do not seek to put a score on the income of each individual but rather to observe the loss of working time of each in the event of an epidemic. This is again a debatable point of view since it is necessary to be able to reproduce the distribution of the work of individuals. At the scale of a society, it seems too difficult to reproduce reality. So the interest of the individual economic evaluation is to put a score by taking a step back at the scale of a population.\\

Note that there are very important elements to observe with this individual evaluation. For example, it will be wise to focus on the cases of children and medical personnel to provide analyses essential to epidemic management, including the stability of the education system.\\

\subsection{Territorial Assessment Scores}

Like the global health assessment, it is necessary to be able to provide illustrative economic scores for each state of society, in order to make the best use of reinforcement learning. With the economy, the focus is on stabilizing production rather than optimizing it. This is why two priority elements must intervene in the overall evaluation : the average economic production of society and the dispersion of production between different spaces and individuals.