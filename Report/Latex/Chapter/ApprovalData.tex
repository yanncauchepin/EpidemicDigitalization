\chapter{Approval with real data}

The idea is to use data that is healthy in relations to humans. It is quite possible to model, reproduce and anticipate reality, only with the help of environmental data. One of the constraints of this is therefore not to use personal data such as GPS tracking. We can find a significant amount of healthy data in a territory :\\

\begin{itemize}
\item The age distribution of the local population.
\item The distribution of household size.
\item Some common affluences of places, depending on calendar parameters.
\item The number of cars per household.
\item The distribution of workers in the different occupational sectors.
\item Mapping data as a whole ; see \textit{OpenStreetMap}.
\item Data on frontier workers.
\item etc.\\
\end{itemize}


This data is intended to be global and requires less memory space. They can cover a large part of the environment.\\

The whole point is to determine the black boxes between them !\\
Finally, there are no checks as one could find in a neural network. There is no test data or validation. But the more healthy data you have, the more intuitive it is to find a link. In this research, the easiest way is to look at the causes and consequences of the real environment ; processes can be identified. We can then run simulations with certain initial conditions and compare with data observed in reality. It is then possible to review little by little the parameters of digitalization to get closer to a reality ; while moving away from personal data and keeping a human ethic.