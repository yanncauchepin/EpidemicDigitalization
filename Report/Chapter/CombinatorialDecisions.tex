\chapter{Combinatorial decisions}

It has not yet been mentioned that several decisions can be taken for the same period. The impacts of decisions can be significantly different and it is therefore possible to add some of them. One example is a decision on the validity of infectious tests, with a curfew from 6 p.m.\\

It might be interesting to consider non-compliance with these rules with a compliance rate in a later version: with each scenario creation, the individual could avoid these constraints according to a probability ; that is, this non- compliance would be applied regardless of the day and the individual, depending on the category of the individual. Of course, the scope of such non-compliance would depend on the seriousness/extent of the decision. But it seems absurd to characterize a subfield of action, which would apply specifically to a temporary niche of a scenario.\\

In the immediate future, the implementation is not constructive enough to focus more on decisions. It would produce aberrant results, which we would seek to optimize...\\

\section{Constraints on individuals}

To apply a specific decision to individuals, one must be interested in the external characteristics of the individual. The idea is to draw a parallel with reality. We cannot therefore use the socio-dynamic classes implemented in the environment. Individuals can then be separated by their ages, known infectious states and dynamic organization. Indeed, individuals whose work is a business can be specifically restricted ; This could then impact several socio-dynamic classes in the implementation for example.\\

\section{Constraints on spaces}

With spaces, however, it seems mandatory to use the categories in the database to apply specific decisions to them. We can then influence this selection by the area of each space to have a better classification in relation to reality. This is the case, for example, when you want to implement decisions on large shopping centres and leave small shops free. It is on the basis of this limitation of distinction that we can arrange the categories. For example, since originally the keys of the openstreetmap file are more detailed, we can redissociate the category of restorations by interior and exterior restorations.\\

\section{Constraints on scenarios}

The constraints on the scenarios are more complex since they themselves depend on static digitalization. They could be reduced to time constraints such as curfew or confinement as well as space selection constraints. Indeed, this is in line with the real case of distance limitations. As a reminder, the space database can filter places itself with a starting location and several criteria.

\section{Constraints on infectious states}

The constraints on infections seem obvious from an epidemic management perspective. In general, they focus on known, not real, conditions, as well as on how to access care, tests and vaccines. We can add other decisions such as the validity of the tests, as a whole to simplify. One of the difficulties is in the predetermination of care, tests and vaccines: we cannot predict these elements at the onset of infection. Perhaps by having a better look at medical processes, we could have an approximation on their onset times and their characteristics ; We can find logics between them. The question then arises about randomness.